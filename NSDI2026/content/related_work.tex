\section{Related Work}\label{sec:related_work}

\subsection{Routing Protocol Suites}

Conventional routing protocol development has largely relied on comprehensive software suites. FRRouting \cite{frrouting2017}, an evolution of Quagga, provides a collection of IP routing protocols for Unix platforms, while XORP \cite{handley2003xorp} offers a programmable routing platform supporting multiple protocols. These suites, while offering stability and wide protocol support, lack the flexibility and modularity required for rapid prototyping of novel routing strategies.

Recent efforts like ONOS~\cite{berde2014onos} have created more programmable environments, but they focus primarily on SDN scenarios rather than distributed routing protocols. xBGP~\cite{wirtgen2023xbgp} enhances BGP with custom code execution, aligning with \textsc{Romam}'s goals of modularity and extensibility. However, xBGP is limited to BGP-specific enhancements, whereas \textsc{Romam} provides a comprehensive framework applicable to a variety of intra-AS routing protocols.


\subsection{SDN and Programmable Networking}

The advent of Software-Defined Networking has introduced new paradigms in network programmability. OpenFlow~\cite{mckeown2008openflow} pioneered the separation of control and data planes, while P4~\cite{bosshart2014p4} enabled protocol-independent packet processors. These approaches primarily focus on centralized control and data plane programmability, which, while powerful, have limitations in distributed environments. Developments such as Intel Tofino chip~\cite{barefoot2018tofino}, have pushed the boundaries of programmable networking hardware. While these efforts have advanced hardware and programming languages, they lack a routing framework for rapid protocol development. \textsc{Romam} fills this gap by combining distributed routing with modular design principles, enabling the creation of adaptive, intelligent routing protocols that can operate effectively in both centralized and distributed environments.

\subsection{Machine Learning in Routing}

Recent years have seen growing interest in applying machine learning to networking problems. Mestres et al. \cite{mestres2017knowledge} introduced the concept of Knowledge-Defined Networking, proposing a new paradigm that incorporates machine learning in network operations. Klaine et al. \cite{klaine2017survey} surveyed machine learning techniques in self-organizing networks, highlighting the potential of AI in this domain.

While these works demonstrate the promise of AI in networking, they often remain theoretical or limited in scope. For instance, RouteNet~\cite{rusek2020routenet} shows promising results in predicting network performance but doesn't provide a framework for implementing ML-driven routing protocols. \textsc{Romam} bridges this gap by providing native support for integrating machine learning algorithms into practical routing decisions, enabling the development and deployment of intelligent, adaptive routing protocols.