\begin{abstract}
The rapid evolution of network applications and environments necessitates advanced, customized routing protocols. However, the lack of comprehensive development and testing platforms often impedes progress in this field. We introduce \textsc{Romam}, an intra-autonomous system (AS) routing framework designed to accelerate the research and development of routing protocols. \textsc{Romam}'s key innovation lies in its modular, highly adaptable framework that integrates both static and dynamic network information, enabling swift prototyping and assessment of advanced routing solutions. We demonstrate \textsc{Romam}'s capabilities through five progressive use cases, spanning from traditional protocols to traffic and QoS-aware approaches. These cases showcase \textsc{Romam}'s versatility in supporting the evolution of routing strategies. In addition, we present a comprehensive monitoring toolchain for pre-deployment evaluation of routing solutions. Our empirical evaluations reveal that \textsc{Romam} can reduce development efforts by up to $90\%$ while providing robust support for advanced routing algorithms. By bridging the gap between theoretical concepts and practical implementation, \textsc{Romam} can be a versatile tool to speed up network routing research, paving the way for next-generation adaptive routing protocols.
\end{abstract}



% An abstract include three key points.
% Highlight the chanllegnes
% Your solution to evolve the challenges.
% the overview of the paper's contribution.


% [Objective] 现状:与本文相关的研究,它们有哪些不足
% [Method] 方法: 我的研究做了什么
% [Result] 结论: 我的结果是什么样的
% [Significance] 意义:我的成果牛在哪里,比之前的人好在哪


% \begin{abstract}
% The proliferation of new applications and network scenarios demands customized and intelligent routing protocols to meet diverse service requirements effectively. However, the absence of robust development and testing platforms often impedes progress in this area.

% To address these challenges, we propose \textsc{Romam}, an intra-autonomous system (AS) routing architecture to accelerate the research and development of intelligent routing protocols. \textsc{Romam} leverages a modular and highly versatile platform that facilitates rapid prototyping and evaluation of new routing solutions. It consists of four main modules: an Information Collection Module for the flexible acquisition of both static topology/link state information and dynamic traffic conditions, a Route Discovery Module to facilitate the discovery of all possible paths in a collaborative, distributed manner, a Traffic Detection Module for detecting and predicting network dynamics, and an Intelligent Forwarding Module for intelligently selecting paths discovered considering cost, network dynamics, and service requirements.

% This paper highlights five diverse use cases that demonstrate the practical benefits of utilizing \textsc{Romam} for routing research. In addition, we introduce a comprehensive monitoring toolchain for evaluating new routing solutions before deployment. Our empirical tests reveal that \textsc{Romam} can reduce coding efforts by up to $90\%$ and offers high compatibility with emerging learning technologies, showcasing its potential to substantially advance the field of network routing.
% \end{abstract}