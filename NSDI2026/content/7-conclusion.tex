\section{Conclusion}\label{sec:conc}

\textsc{Romam} is a platform designed to facilitate the development of intelligent distributed routing protocols. By integrating modular components that foster the adoption of advanced routing algorithms and traffic management strategies, \textsc{Romam} offers substantial benefits to network researchers and developers, enabling them to efficiently develop and deploy sophisticated customized routing protocols. In addition, it provides a modularized library that simplifies the customization and extension of routing capabilities, making it easier to adapt to varying network requirements and conditions.

We have showcased the capability of \textsc{Romam} by implementing five routing protocols, underscoring its and effectiveness as a research tool. Looking forward, \textsc{Romam} is poised to play a crucial role in the intersection of artificial intelligence and network management, potentially driving the evolution of next-generation intelligent routing technologies.

\textsc{Romam} would be more useful if it can be test using real-world network traffic data. Currently, \textsc{Romam} relies on ns-3 for generating network traffic, using random generators for TCP/UDP traffic. This approach falls short of capture the complexity and variability of actual network traffic, creating a gap that may limit our ability to obtain realistic network state. Given the promising results, our future plans include implementing \textsc{Romam} in re-programmable routers.