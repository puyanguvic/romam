\section{Introduction}\label{intro}

Traditional Internet routing, designed primarily for simplicity and scalability, aimed to deliver packets in a best-effort manner \cite{huitema1995routing}. Today's network demands, driven by emerging applications such as AR/VR/XR and real-time control, require different dimensions such as ultra-high reliability, ultra-low latency, and minimal jitter \cite{dang2020should}. These requirements exceed the capabilities of conventional routing protocols. Moreover, modern networks are increasingly heterogeneous and dynamic, encompassing land, air, sea, and space domains, utilizing diverse networking devices from ground to satellites \cite{liu2018space}. This complexity introduces new challenges, making traditional ``one-size-fits-all" routing protocols inadequate.

In response, large network operators have begun developing customized routing protocols tailored to specific environments, such as Software Defined Networks (SDNs) \cite{jouppi2017datacenter, shao2021accessing, kalia2019datacenter, koch2023painter} for data centers and Content Delivery Networks (CDNs) \cite{chen2023darwin, song2020learning} for efficient content distribution. While these specialized protocols address niche needs effectively, they lack the flexibility to be applied across varied network types and scenarios.

Furthermore, routing protocol development is hampered by the inherent complexity, high cost, and limitations of existing development frameworks. While SDN frameworks like OpenFlow \cite{mckeown2008openflow} and P4 \cite{bosshart2014p4} offer powerful capabilities for programming the data plane and centralizing network control, they were not specifically designed to support the development of distributed routing protocols. These frameworks focus on network-wide programmability rather than providing a structured approach for designing and implementing new distributed routing algorithms. Similarly, platforms such as FRRouting \cite{frrouting2017}, Bird \cite{bird}, and OpenWRT \cite{openwrt}, designed with legacy systems in mind, do not provide the tools or architectural support necessary to develop intelligent, adaptive routing protocols that can respond autonomously to changing network conditions and user demands.

To address these challenges and support the rapid evolution of network technologies, we present \textbf{\textsc{Romam}}\footnote{\textsc{Romam} is the accusative of direction in Latin meaning ``to Rome", symbolizing our framework's goal of exploring multiple routing possibilities, much like the diverse roads that once led to Rome.}~(ROuting MAnage Modules), a novel intra-autonomous system (AS) routing framework. By modularizing the routing protocol development process, \textsc{Romam} not only simplifies the creation of advanced routing algorithms but also enhances the adaptability of network operations to dynamic conditions, significantly improving packet-level Quality of Service (QoS). The architecture supports a flexible assembly of its components, allowing for customized solutions that address the specific requirements of modern network environments.

We demonstrate the effectiveness of \textsc{Romam} through the implementation of five routing protocols in ns-3, showcasing substantial improvements in service quality and network efficiency. Our contributions include:

\begin{itemize}
    \item We present \textsc{Romam}, a novel modular framework for developing routing protocols. This framework decomposes routing functions into reusable components, allowing rapid prototyping and development of different routing strategies. By leveraging this modular design, \textsc{Romam} reduces coding effort by up to $90\%$ compared to traditional approaches, significantly streamlining the protocol development process.
    
    \item We develop an extensive library of routing algorithms, queue disciplines, and traffic detection methods within the \textsc{Romam} framework. This comprehensive library serves as a foundation for researchers and developers to quickly experiment with and innovate new routing protocols, thereby accelerating the network routing research and development process.
    
    \item \textsc{Romam} integrates static network topology information with real-time traffic data, enabling the development of adaptive routing protocols. This integration allows protocols to effectively respond to changing network conditions, which is critical for improving packet-level Quality of Service (QoS) in dynamic network environments.
    
    \item We demonstrate the capabilities of \textsc{Romam} by implementing five different routing protocols, including two traditional and three innovative designs. These implementations showcase \textsc{Romam}'s flexibility in supporting a wide range of routing strategies, from basic approaches to highly sophisticated, adaptive protocols, illustrating its versatility in addressing various network routing challenges.
\end{itemize}

The rest of the paper is organized as follows: Sec.~\ref{sec:principle} outlines the design principles of \textsc{Romam}. Sec.~\ref{sec:D&I} details the architecture and implementation methodology. Sec.~\ref{sec:usecases} presents five use cases demonstrating \textsc{Romam}'s capabilities. Sec.~\ref{sec:evaluation} provides a comprehensive evaluation of \textsc{Romam}'s performance. Sec.~\ref{sec:related_work} discusses related work, and Sec.~\ref{sec:conc} concludes the paper with insights into future directions.

\textsc{Romam} is open source, and all codes are anonymously available at \url{https://anonymous.4open.science/r/romam-7BC0/}.