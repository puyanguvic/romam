\section{Discussion, limitations and future work}
\textbf{Improving \textsc{Romam}.} \textsc{Romam} would be more useful if it can be test using real-world network traffic data. Currently, \textsc{Romam} relies on ns-3 to generate network traffic, utilizing random generators for TCP/UDP traffic. This approach falls short of replicating the complexity and variability of actual network traffic, presenting a gap that can restrict our ability to obtain realistic network state datasets. Consequently, this limitation impacts the training and validation of intelligent routing protocols.


In addition, implementing \textsc{Romam} to the Linux kernel stack would boost the framework's portability and applicability. At present, \textsc{Romam} is configured to operate on the ns-3 network simulator, tailored specifically to the ns-3 network stack. Aligning \textsc{Romam}'s interfaces with those of the Linux kernel would allow \textsc{Romam} to be ported from network simulators to Docker emulators like mininet, and to real network devices. This remains a future research issue. 


\textbf{Enhancing Intelligent Protocol Design.} The Octopus protocol serves as an illustrative case study of the integration of light-weight learning techniques into routing protocol forwarding within the \textsc{Romam} architecture. This example highlights the potential of learning technologies in enhancing distributed routing protocols. The use of an MAB algorithm allows Octopus to optimize forwarding decisions based on historical cumulative loss data, effectively guiding the protocol toward selecting routes that offer greater benefits. This approach not only improves the efficiency of data flow across the network but also demonstrates how learning algorithms can dynamically adapt to changing network conditions to optimize path selections continuously.

%In the evaluation of routing protocols, two primary indicators are crucial: the on-time delivery ratio and the delay budget efficiency. The former measures the proportion of packets that reach their destination prior to their deadline. In real-time video streaming and online gaming, for instance, a high on-time delivery ratio is essential for ensuring a satisfactory user experience. A higher ratio typically corresponds to lower packet loss and better network performance. Hence, our focus lies in enhancing 
How to Octopus, aiming to bolster both its per-packet delay performance and convergence rate, remains an open issue. This endeavor entails the design of more purposeful reward functions capable of leveraging all available information to drive improvements, as well as a thorough redesign of the initialization procedures.

\textbf{Compatibility of Multipath Routing and Transport Layer Congestion Control Algorithms.} One important issue with multi-path routing is packet reordering, which can occur when routing decisions are made in a stateless manner (i.e., without considering flow identifiers). This reordering can trigger unnecessary congestion control measures in the transport layer. Although this problem is beyond the scope of this work, we predict that future transport layer protocols can be more intelligent in detecting network congestion without relying on duplicated ACKs, so networks can fully enjoy the benefits of multipath routing.